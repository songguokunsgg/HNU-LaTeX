% !Mode:: "TeX:UTF-8" 
\cleardoublepage
\chapter{图表公式排版}%{Figures, Tables and Equations}

为了方便大家使用本模板撰写学位论文,本章对论文写作中经常用到的{\hei 图、表、公式}等内容的排版方法做一个简单的介绍。

%=========================================================================================
\section{图}%{Figures}

插图须紧跟文述。在正文中,一般应先见图号及图的内容后再见图,一般情况下不能提前见图,特殊情况须延后的插图不应跨节。

\LaTeX{} 中所使用的图片通常为~PDF~格式,图片应大小适宜,主题明确,层次清楚,金相组织类的照片一定要有比例尺。

图应具有“自明性”,即只看图、图题和图例,不阅读正文,就可理解图意。
图中的标目是说明坐标轴物理意义的项目,它是由物理量的符号或名称和相应的单位组成。物理量的符号由斜体字母标注,单位的符号使用正体字母标注,量与单位间用斜线隔开。
%-----------------------------------------------------------------------------------------
\subsection{单幅图}%{Single Figure}

\begin{figure}[!ht]
	\centering
	\includegraphics[width=0.8\textwidth]{chapter2/exam1.PNG}
	\caption{windows 壁纸} \label{fig:wallpaperOfWindows}
\end{figure}

% 建议规范命令,方便后续查找
引用这一幅图 \ref{fig:wallpaperOfWindows}

%=========================================================================================
\section{表}%{Tables} 

表格的设计应紧跟文述。表的编排一般是内容和测试项目由左至右横读,数据依序竖读,应有自明性。若为大表或作为工具使用的表格,可作为附表在附录中给出,论文中的表格参数应标明量和单位的符号。

表中各物理量及量纲均按国际标准(SI) 及国家规定的法定符号和法定计量单位标注。

\begin{table}[!ht]
	\renewcommand{\arraystretch}{1.2}
	\centering
	% small 不能省略
	\small
	\caption{表题也是五号字}
	\label{tbl:tableOne}
	\begin{tabularx}{\textwidth}{*{4}Y}
		\toprule[2pt]
		组号 & DOA / $^\circ$ & 带宽 / MHz & INR / dB \\
		\midrule[0.25pt]
		1  & $-30$          & 20       & 60       \\
		2  & 20             & 10       & 50       \\
		3  & 40             & 5        & 40       \\
		\bottomrule[2pt]
	\end{tabularx}
\end{table}



%=========================================================================================
\section{公式}%{Equations}

在 \LaTeX{} 中,行内公式用~$\$\ \ \$$~符号括起来。行间公式应另起一行,居中编排,较长的公式尽可能在等号后换行,或者在“+”、“-”等符号后换行。公式中分数线的横线,长短要分清,主要的横线应与等号取平。

公式后应注明编号,公式号应置于小括号中。写在右边行末,中间不加虚线。

公式下面的“式中:”两字左起顶格编排,后接符号及其解释;解释顺序为先左后右,先上后下;解释与解释之间用“;”隔开。

公式中各物理量及量纲均按国际标准(SI)及国家规定的法定符号和法定计量单位标注,禁止使用已废弃的符号和计量单位。

%-----------------------------------------------------------------------------------------
\subsection{单个公式}%{Equations}

\LaTeX{} 最强大的地方在于对数学公式的编辑,不仅美观,而且高效。单个公式的编号如式 (\ref{eqt:singleFormula}) 所示,该式是正态分布的概率密度函数,
\begin{equation}\label{eqt:singleFormula}
	f_Z(z) = \frac{1}{\pi\sigma^2} \exp\left(-\frac{|z-\mu|^2}{\sigma^2}\right)
\end{equation}
式中:$\mu$ 是 Gauss 随机变量 $Z$ 的均值;$\sigma^2$ 是 $Z$ 的方差。

%-----------------------------------------------------------------------------------------
\subsection{多个公式}%{Subequations}

多个公式作为一个整体可以进行二级编号,如式 (\ref{eqt:multiFormula}) 所示,该式是连续时间 Fourier 变换的正反变换公式,
\begin{subequations}\label{eqt:multiFormula}
	\begin{align}
		X(f) & = \int_{-\infty}^{\infty}x(t)e^{-j2\pi f t}\dif t \\
		x(t) & = \int_{-\infty}^{\infty}X(f)e^{j2\pi f t}\dif f
	\end{align}
\end{subequations}
式中:$x(t)$ 是信号的时域波形;$X(f)$ 是 $x(t)$ 的 Fourier 变换。

如果公式中包含推导步骤,可以只对最终的公式进行编号,例如:
\begin{align}
	\mbf{w}_{\mathrm{smi}} & = \alpha \left[\frac{1}{\sigma_n^2}\mbf{v}(\theta_0) - \frac{1}{\sigma_n^2}\mbf{v}(\theta_0) + \sum_{i=1}^{N} \frac{\mbf{u}_i^H\mbf{v}(\theta_0)}{\lambda_i} \mbf{u}_i\right] \nonumber                         \\
	                       & = \frac{\alpha}{\sigma_n^2} \left[\mbf{v}(\theta_0) - \sum_{i=1}^{N}\mbf{u}_i^H\mbf{v}(\theta_0)\mbf{u}_i +  \sum_{i=1}^{N}\frac{\sigma_n^2\mbf{u}_i^H\mbf{v}(\theta_0)}{\lambda_i} \mbf{u}_i \right] \nonumber \\
	                       & = \frac{\alpha}{\sigma_n^2} \left[\mbf{v}(\theta_0) - \sum_{i=1}^{N} \frac{\lambda_i-\sigma_n^2}{\lambda_i} \mbf{u}_i^H\mbf{v}(\theta_0)\mbf{u}_i \right]
\end{align}

\subsection{算法}

\subsection{A2}
算法描述使用 algorithm2e 宏包,效果如算法 \ref{alg:examDescription} 所示。

\begin{algorithm}[H]
    \caption{示例算法详细描述}
    \label{alg:examDescription}
    \KwIn{$\mbf{x}(k), \quad \mu, \quad \mbf{w}(0)$}
    \KwOut{$y(k), \quad \varepsilon(k)$}
    %
    \For{$k=0,1,\cdots$}{
        $y(k) = \mbf{w}^H(k)\mbf{x}(k)$ \tcp*{output signal}\;
    }
	\While{i < 10}{
		i++\;
	}

	\If{i<10}{
		break\;
	}
\end{algorithm}

源代码使用 listings 宏包,LMS~算法的 Verilog 模块端口声明如代码 \ref{code_ch3_lms} 所示。

{\zihao{5}
    \begin{lstlisting}[caption={空时~LMS~算法~Verilog~模块端口声明},label={code_ch3_lms}]
    module stap_lms
    #(
    parameter      M                = 4,    // number of antennas
                   L                = 5,    // length of FIR filter
                   W_IN             = 18,   // wordlength of input data
                   W_OUT            = 18,   // wordlength of output data
                   W_COEF           = 20    // wordlength of weights
    )(
    output  signed [W_OUT-1:0]      y_i,    // in-phase component of STAP output
    output  signed [W_OUT-1:0]      y_q,    // quadrature component of STAP output
    output                          vout,   // data valid flag of output (high)
    input          [M*W_IN-1:0]     u_i,    // in-phase component of M antennas
    input          [M*W_IN-1:0]     u_q,    // quadrature component of M antennas
    input                           vin,    // data valid flag for input (high)
    input                           clk,    // clock signal
    input                           rst     // reset signal (high)
    );
    \end{lstlisting}
}

\subsection{数学环境}

主要为 定义、定理、命题、证明 四个:

\begin{Definition}
	这是一个定义。
\end{Definition}

\begin{Proposition}
	这是一个命题。
\end{Proposition}

\begin{Theorem}
	这是一个定理。
\end{Theorem}

% 必须手动编号
\begin{Proof}
	1. 证明 1。

	2. 证明 2。
\end{Proof}